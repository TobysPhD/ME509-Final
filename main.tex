\documentclass[conf]{new-aiaa}
%\documentclass[journal]{new-aiaa} for journal papers
\usepackage[utf8]{inputenc}

\usepackage{graphicx}
\usepackage{amsmath}
\usepackage[makeroom]{cancel}
\usepackage[version=4]{mhchem}
\usepackage{siunitx}
\usepackage{longtable,tabularx}
\usepackage[framed,numbered,autolinebreaks,useliterate]{mcode}
\setlength\LTleft{0pt} 
\usepackage{gensymb}
\usepackage{rotating}
\usepackage{hyperref}
\title{ME509 Heat Transfer Term Project}

\author{Tobin A. Nelson}
\affil{University of Alabama, Tuscaloosa, AL, 35487}


\begin{document}
\maketitle


\section{Instructions}
\lettrine{T}his project asks students to modify two provided MATLAB codes which utilize the Crank Nicolson technique to model one dimensional heat transfer in a flat plate with time dependent heat flux at x = 0, perfectly insulated at x=L, and an initial temperature of 0 across the plate (X22B-0T0). The students project descriptions asks for the code to be modified to create a Type 3 (Robins) boundary condition at x=0 and t Type 2 at x=L. The first boundary condition being characterized by $Bi_L=hL/k$ and an environment temperature $T_\infty [^oC]$. The following parameters were also given: L=2, nx=40, dt=0.01, T\textunderscore inf=5, x=[0:L/nx:L], t\textunderscore final=1.8, and $q(t)=1.0 W/m^2$ beginning at t=0. 
\section{Theory}
To modify the code, the proper form of the heat equation must be found and then discretized using the Crank-Nicolson method.
\subsection{The Heat Equation}
The given modifications indicate one dimensional transient heat transfer in a thin slab with convection at x=0 and heat conduction at x=L. This can be characterized by X32B1-T0 with the following governing equation and boundary and initial conditions.

\begin{equation}
\label{sample:equation}
   \frac{\partial^2 T}{\partial x^2}=\frac{1}{\alpha}\frac{\partial T}{\partial t}
\end{equation}

\begin{equation}
\label{sample:equation}
  \left[ -k\frac{\partial T}{\partial x}+hT\right]_{x=0}=hT_{\infty}
\end{equation}

\begin{equation}
\label{sample:equation}
  \left[ -k\frac{\partial T}{\partial x}\right]_{x=L}=q(t)
\end{equation}

\begin{equation}
\label{sample:equation}
  T(0)=0
\end{equation}

\subsection{Crank-Nicolson}
The Crank Nicolson method is a second order in space and time, finite differencing method. It works by combining the forward Euler method at point n with the backward Euler method at point n+1. This combines the simplicity of the forward Euler with the stability of the backward Euler method, making it unconditionally stable. It can be expressed as follows:
\begin{equation}
\label{sample:equation}
\frac{\partial T}{\partial t}=\alpha \frac{1}{2}\left[\frac{\partial^2 T}{\partial x^2}^{t=j}+\frac{\partial^2 T}{\partial x^2}^{t=j+1}\right]
\end{equation}
Applying finite differencing to this yields:
\begin{equation}
\label{sample:equation}
\frac{T_i^{j+1}-T_i^j}{\Delta t}=\alpha \frac{1}{2}\left[\frac{T_{i+1}^j-2T_i^j+T_{i-1}^j}{\Delta x^2}+\frac{T_{i+1}^{j+1}-2T_i^{j+1}+T_{i-1}^{j+1}}{\Delta x^2}\right]
\end{equation}

Using the cell-dependent Fourier Number $Fo=\frac{\alpha \Delta t}{\Delta x^2}$ and rearranging, the following implicit/explicit expression for the temperature at position i and time j+1 can be reached:

\begin{equation}
\label{sample:equation}
  T_i^{j+1}-\frac{Fo}{2}\left[T_{i+1}^{j+1}-2T_i^{j+1}+T_{i-1}^{j+1}\right]=T_i^j+\frac{Fo}{2}\left[T_{i+1}^j-2T_i^j+T_{i-1}^j\right]
\end{equation}

This can be solved by expressing the previous equation as matrix function:

\begin{equation}
\label{sample:equation}
\bf{A} \cdot \bf{T}^{j+1}=\bf{B} \implies \bf{T}^{j+1} =\bf{A}^T \cdot \bf{B}
\end{equation}
Or: 
\begin{equation}
\label{sample:equation}
\begin{split}
\begin{bmatrix}
    &       &       & \cdots&       &       \\
-Fo &2(1+Fo)&-Fo    & 0     &\cdots &       \\
0   &-Fo    &2(1+Fo)&-Fo    & 0     &\cdots \\
    &       &       & \ddots&       &       \\
    &\cdots & 0     &-Fo    &2(1+Fo)&-Fo    \\
    &       &       & \cdots&       &   
\end{bmatrix} 
\cdot
\begin{bmatrix}
T_1^{j+1} \\
T_2^{j+1} \\
\\
\vdots\\
\\
T_N^{j+1} \\
\end{bmatrix} 
 \cr =
 \begin{bmatrix}
        &               &               & \cdots        &                   &           \\
FoT_1^j &2(1-Fo)T_2^j   &FoT_3^j        & 0             &\cdots             &           \\
0       &FoT_2^j        &2(1-Fo)T_3^j   &FoT_4^j        & 0                 &\cdots     \\
        &               &               & \ddots        &                   &           \\
        &\cdots         & 0             &Fo T_{N-2}^j   &2(1-Fo)T_{N-1}^j   &FoT_{N}^j  \\
        &               &               & \cdots        &                   &   
\end{bmatrix} 
\end{split}
\end{equation}

\subsection{Boundary Conditions} Now that the matrices have been built to solve for the interior nodes, the boundary conditions need to be handled. To do this, the energy balances at the first and last node will be calculated using the following:

\begin{equation}
\label{sample:equation}
   q_{1-}+q_{1+}=\frac{\rho c \Delta x}{2}\frac{\partial T_1}{\partial t}
\end{equation}

For the first node, heat transfer from the left is in the form of convection, $q_{1-}=-h(T_1-T_\infty)$, and from the right side is conduction with node 2, $q_{1+}=-k(T_1-T_2)/\Delta x$. The energy balance for node 1 can now be written.

\begin{equation}
\label{sample:equation}
h(T_\infty-T_1) + k\frac{(T_2-T_1)}{\Delta x}=\frac{\rho c \Delta x}{2}\frac{\partial T_1}{\partial t}
\end{equation}

Now the Crank Nicolson implicit/explicit method can be applied to the spatial differences and the equation discretized with respect to time.

\begin{equation}
\label{sample:equation}
\frac{h}{2}(T^j_\infty-T^j_1+T^{j+1}_\infty-T^{j+1}_1) + \frac{k}{2}\frac{(T^{j}_2-T^{j}_1+T^{j+1}_2-T^{j+1}_1)}{\Delta x}=\frac{\rho c \Delta x}{2}\frac{T_1^{j+1}-T^{j}_1}{\Delta t}
\end{equation}

\begin{equation}
\label{sample:equation}
\frac{h \Delta x}{k}(T^j_\infty-T^j_1+T^{j+1}_\infty-T^{j+1}_1) +(T^{j}_2-T^{j}_1+T^{j+1}_2-T^{j+1}_1)=\frac{\rho c \Delta x^2}{k \Delta t}(T_1^{j+1}-T^{j}_1)
\end{equation}

\begin{equation}
\label{sample:equation}
\frac{k \Delta t}{\rho c \Delta x^2}\frac{h \Delta x}{k}(T^{j+1}_\infty-T^{j+1}_1)
+\frac{k \Delta t}{\rho c \Delta x^2}(T_2^{j+1}-T_1^{j+1})-T_1^{j+1}
=
\frac{k \Delta t}{\rho c \Delta x^2}\frac{h \Delta x}{k}(T^{j}_1-T^{j}_\infty)
+\frac{k \Delta t}{\rho c \Delta x^2}(T_1^{j}-T_2^{j})-T_1^{j}
\end{equation}

Now the equation will be non-dimensionalized using $\tilde{q}=q/q_{ref}$, $\tilde{T}=T k/q_{ref}$, $\tilde{t}=\alpha t/L^2$, $\tilde{x}=x/L$, and $Bi_L=hL/k$. Additionally, $T_\infty$ is constant, so $T_\infty^j=T_infty^{j+1}=T_\infty$.

\begin{equation}
\label{sample:equation}
Fo Bi \Delta \tilde{x} \tilde{T}^{j+1}_1
+Fo (\tilde{T}_1^{j+1}-\tilde{T}_2^{j+1})-\tilde{T}_1^{j+1}
=
Fo Bi \Delta \tilde{x} (2\tilde{T}_\infty-\tilde{T}^{j}_1)
+Fo (\tilde{T}_2^{j}-\tilde{T}_1^{j})+\tilde{T}_1^{j}
\end{equation}

\begin{equation}
\label{sample:equation}
[Fo(Bi \Delta \tilde{x}+1)+1] (\tilde{T}^{j+1}_1)
-Fo \tilde{T}_2^{j+1}
=
Fo Bi \Delta \tilde{x} (2\tilde{T}_\infty-\tilde{T}^{j}_1)
+Fo (\tilde{T}_2^{j}-\tilde{T}_1^{j})+\tilde{T}_1^{j}
\end{equation}

This can now be used to create the first line of the matrix.
\\
The boundary condition at x=L is a conductive boundary where the heat flow varies with time. To find the energy balance, Equation 10 will be modified for node N.

\begin{equation}
\label{sample:equation}
   q_{N-}+q_{N+}=\frac{\rho c \Delta x}{2}\frac{\partial T_N}{\partial t}
\end{equation}

The heat flow from the left is $-k(T_N-T_{N-1})/\Delta x$, and the heat flow from the right is q(t).

\begin{equation}
\label{sample:equation}
   k\frac{(T_{N-1}-T_N)}{\Delta x}+q(t)=\frac{\rho c \Delta x}{2}\frac{\partial T_N}{\partial t}
\end{equation}

A non-dimensional equation can be derrived using the same process as for node 1.

\begin{equation}
\label{sample:equation}
   \frac{k}{2}\frac{(T^j_{N-1}-T^j_N+T^{j-1}_{N-1}-T^{j-1}_N)}{\Delta x}+\frac{1}{2}(q^j+q^{j-1})=\frac{\rho c \Delta x}{2}\frac{\partial T_N}{\partial t}
\end{equation}

\begin{equation}
\label{sample:equation}
   \frac{k \Delta t}{\rho c \Delta x^2}(T^j_{N-1}-T^j_N+T^{j-1}_{N-1}-T^{j-1}_N)+\frac{\Delta t}{\rho c \Delta x}(q^j+q^{j-1})=T^{j+1}_N-T^{j}_N
\end{equation}

\begin{equation}
\label{sample:equation}
   Fo (\tilde{T}^j_{N-1}-\tilde{T}^j_N+\tilde{T}^{j+1}_{N-1}-\tilde{T}^{j+1}_N)+Fo \Delta x (\tilde{q}^j+\tilde{q}^{j+1})=\tilde{T}^{j+1}_N-\tilde{T}^{j}_N
\end{equation}

\begin{equation}
\label{sample:equation}
   -Fo (\tilde{T}^{j+1}_{N-1}-\tilde{T}^{j+1}_N)+\tilde{T}^{j+1}_N=Fo (\tilde{T}^j_{N-1}-\tilde{T}^j_N)+\tilde{T}^{j}_N+Fo \Delta x (\tilde{q}^j+\tilde{q}^{j+1})
\end{equation}

\begin{equation}
\label{sample:equation}
   (Fo+1)\tilde{T}^{j+1}_N-Fo\tilde{T}^{j+1}_{N-1}=Fo (\tilde{T}^j_{N-1}-\tilde{T}^j_N)+\tilde{T}^{j}_N+Fo \Delta x (\tilde{q}^j+\tilde{q}^{j+1})
\end{equation}

\section{Results}
These changes were made to the two provided codes, which can be seen in Appendix A. It was then run with a Biot numbers of 0, 20, and 100 and plots of T vs. x were generated for t = 0.1, 0.45, 0.9, 1.35, and 1.8 s. These can be seen in Figures 1-3.
\\
\begin{figure}[hbt!]
\centering
\includegraphics[width=.5\textwidth]{Bi0.jpg}
\caption{Results at $Bi_L$=0}
\end{figure}
\\
The results for the Biot number of 0 show that it behaves as if there is an adiabatic boundary at x = 0. Which could be predicted from understanding that convective transfer goes to 0 as the Biot number decreases, and from looking at equation 16. It can also be seen that early on, the left side stays at zero and starts to raise above zero and towards steady state once the diffusion reaches the left side and time advances.
\\
\begin{figure}[hbt!]
\centering
\includegraphics[width=.5\textwidth]{Bi20.jpg}
\caption{Results at $Bi_L$=20}
\end{figure}
\\
Looking at the results for a Biot number of 20, it can be seen that at the start most of the heat transfer is coming from the convective term and that the temperature at x=0 reaches the steady state temperature ($T_\infty$) quickly. However, the system does not reach steady state until 1.8 s. It can also be seen that the system balances out with the temperature at x=L slightly above that at x=0. This is to be expected since the boundary for x=L specifies a constant rate of heat flow into the system. This means that while at the start, the left boundary experiences convective heating and dominates the heat transfer rates into the system, after the system reaches steady state it becomes a convective cooling boundary to allow the release of the heat entering from the right boundary.
\begin{figure}[hbt!]
\centering
\includegraphics[width=.5\textwidth]{Bi100.jpg}
\caption{Results at $Bi_L$=100}
\end{figure}
\\
The results for the Biot number of 100 shows that the higher rate of heat transfer quickly brings the sytem up to temperature. It can also be seen that once the entire system has reached that of the free stream convective temperature on the left, the only heat into the system comes from the right boundary resulting in slower temperature increases.

\clearpage
\section*{Appendix A: MATLAB Code}

\begin{lstlisting}
close all
clear all
clc
m=0;
% script to run CN
% 
% space and time parameters
L=2;
nx = 40; % # segments
dt_tilde = 0.001; % time step (dimensionless)
x_tilde = [0:1/nx:1]; % x/L dimensionless

% reference parameters 
q_ref=10;
k=100;
T_ref=q_ref*L/k;
T_inf=5/T_ref;
T_tilde = zeros(nx+1,1);
t_final = 1.8; % given
time = [0 : dt_tilde : t_final ];
t_tilde = time/t_final;
q_tilde_0 = ones(length(time)); % given
nt = floor( t_final / dt_tilde );
for Bi=[0, 20, 100]
m=m+1;
% figure(1)
% plot( t_tilde, q_tilde_0,'LineWidth',3  );
% ax = gca;
% ax.FontSize = 30; 
% ylabel('Dimensionless heat flux','FontSize',30)
% xlabel('Dimensionless time','FontSize',30)
% title('Applied heat (dimensionless)')

T_tilde_save = zeros( length(t_tilde), nx+1 );
T_tilde_save(1, :) = T_tilde'; % save the initial condition
for j = 1 : length(t_tilde)-1
    T_tilde = fCrankNicolson2( nx, dt_tilde, T_tilde, Bi, T_inf, q_tilde_0(j), q_tilde_0(j+1));
    T_tilde_save(j+1, :) = T_tilde';
end

times=t_final/dt_tilde*[0.1, 0.45, 0.9, 1.35, 1.8]/1.8;
figure(m)
plot(x_tilde,T_tilde_save(times,:))
title('Dimensionless Temperature vs. Distance for Bi = ', Bi)
xlabel('$\tilde{x}$','Interpreter','latex')
ylabel('$\tilde{T}$','Interpreter','latex')
legend('0.1 s', '0.45 s', '0.9 s', '1.35 s', '1.8 s','Location','southoutside','Orientation','horizontal')
% figure(2)
% surf( x_tilde, t_tilde(1:nt-1), T_tilde_save(1:nt-1,:) );
% ax = gca;
% ax.FontSize = 30; 
% % xlabel('x_{tilde}','FontSize',30)
% % ylabel('Dimensionless time','FontSize',30)
% % zlabel('Dimensionless temp','FontSize',30)
% 
% figure(3)
% plot( t_tilde(1:nt-1), T_tilde_save( 1:nt-1, 1 ),'LineWidth',3  );
% ax = gca;
% ax.FontSize = 30; 
% % ylabel('Dimensionless temp','FontSize',30)
% % xlabel('Dimensionless time','FontSize',30)
end


function [T_tilde_new] = fCrankNicolson2(nx, dt_tilde, T_tilde_old, Bi, T_inf, q_tilde_j, q_tilde_jp1)
%
% Crank Nicolson routine to advance solution from t_j to t_jp1 (which is t_j+1)
%  X22B-0T0  q(t) = arbitrary function of time 
% constant properties
% nx = number of segments
%T_tilde_old = "old" temperature, dimensionless
% q_tilde_j = heat flux at the boundary at the current time
% q_tilde_jpq = heat flux at the boundary at the current time + dt_tilde
% dimensionless variables dt_tilde = alpha*dt/L^2 (time step)
%               computed  dx_tilde = L/nx = 1/nx
%                          T_tilde = T / (q_ref*L/k)
%                          q = q/q_ref
dx_tilde = 1/nx;
Fo = dt_tilde * nx^2;  % cell Fourier number, local dimensionless time = alpha*dt/dx^2, L = 1 nondimensionalized
Amat = zeros( nx+1, nx+1 );
RHS = zeros( nx+1, 1);
%

% fill the internal nodes first
for i = 2 : nx
    Amat(i,i-1) = -Fo/2;
    Amat(i,i) = 1 + Fo;
    Amat(i,i+1) = -Fo/2;
    RHS(i) = Fo/2 * (T_tilde_old(i-1)+T_tilde_old(i+1)) + (1-Fo) * T_tilde_old(i);
end

% boundary nodes
Amat(nx+1,nx) = -Fo;
Amat(nx+1, nx+1) = Fo + 1;
RHS(nx+1) = Fo*T_tilde_old(nx) + T_tilde_old(nx+1)*(1-Fo) + (q_tilde_j + q_tilde_jp1)*Fo/nx;

Amat(1,1) = (Fo*(Bi/nx+1)+1);
Amat(1,2) = -Fo;
RHS(1) = Fo*Bi/nx*(2*T_inf-T_tilde_old(1))+Fo*(T_tilde_old(2)-T_tilde_old(1))+T_tilde_old(1);
T_tilde_new = Amat \ RHS;  % "\" does something like forward elim/Back subst.
end




\end{lstlisting}
\end{document}
